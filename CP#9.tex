
\documentclass[12pt]{article}

\usepackage{amsmath}

\begin{document}
	\title{Respuesta de la Clase Pr\'actica 9}
	\author{Daniel De La Cruz Prieto}
	\date{\today}
	
	\maketitle
	
	
	\section*{Ejercicio 1}
	
	\paragraph{Respuesta I-a }
	
    Calcular los valores de $a$ y de $b$ si se sabe que $EX = 3.8.$
    
	\begin{equation*}
		P_X\left(2\right) = \sum_{j=1}^{\infty}	P_{\left(X,Y\right)} \left(x,b_j\right) = \sum_{j=1}^{\infty} P\left(X=2,Y=b_j\right)	
	\end{equation*}
	
	\begin{equation*}
		\begin{array}{rcccccccl}
		P_X\left(2\right) & = & 0.1& +& 0 &+& 0.3 & = & 0.4
		\\
		P_X\left(4\right) & = & a &+& 0.2 &+ & 0 & = & a + 0.2
		\\
	    P_X\left(6\right) & = & 0& +& 0.2& + & b & = & b + 0.2
		\end{array}
	\end{equation*}
	
	\begin{equation*}
	    EX = \sum_{i=1}^{\infty} x_j P\left(X=x_j\right)
    \end{equation*}
    
	\begin{equation*}
	  \begin{array}{rcccccl}
	  EX & = & 2(0.4)		   & + & 4(a+0.2)             & + & 6(b+0.2)
	  \\
	  \\
	  EX & = & 2 * \frac{2}{5} & + & 4a + 4 * \frac{1}{5} & + & 6b + 6 * \frac{1}{5}	
	  \\
	  \\
	  EX & = & \frac{4}{5}     & + & 4a + \frac{4}{5}     & + & 6b + \frac{6}{5}
	  \\
	  \\
	  EX & = & \frac{14}{5}   & + & 4a                    & + & 6b 
	  \\
	  \\
	  \Rightarrow 1 & = & 4a & + & 6b  
	  \end{array}
	\end{equation*}
	
	Se sabe que :
	
	
	\begin{equation*}
		\sum_{i=1}^{\infty} \sum_{j=1}^{\infty}	P_{\left(X,Y\right)} \left(a_i,b_j\right) = 1    
	\end{equation*}
	\begin{equation*}
	\begin{array}{rcl}
		\sum_{i=1}^{\infty} \sum_{j=1}^{\infty}	P_{\left(X,Y\right)} \left(a_i,b_j\right) & = & 0.1 + 0.3 + a + 0.2 + 0.2 + b 
		\\
	        	1            & = & 0.8 + a + b 
	    \\
	         1 - 0.8         & = & a + b
	    \\
	           0.2           & = & a + b
	    \\
	      \frac{1}{5}        & = & a + b
		\end{array}  
	\end{equation*}

    Podemos plantear el sistema siguiente y obtenener los valores de a y b :
    \begin{equation*}
    \begin{array}{rccccl}
    I & 4a & + & 6b & = & 1 
    \\
    \amalg & a & + & b & = & \frac{1}{5}
    \end{array}
    \end{equation*}
    
    Y sustituyendo en I :
    
    \begin{equation*}
    \begin{array}{rcl}
    4(\frac{1}{5} - b)  +  6b & = & 1
    \\
    \\
    \frac{4}{5} - 4b  +  6b & = & 1
    \\ 
    \\
    \frac{4}{5}  +  2b & = & 1
    \\
    \\
    2b & = & 1  -  \frac{4}{5}
    \\
    \\
    2b & = & \frac{1}{5}
    \\
    \\
    b & = & \frac{1}{10}
    \end{array}    
    \end{equation*}
    
    
    Despejando $b = \frac{1}{10}$ en la ecuaci\'on $\amalg$ para hallar a :
    
    
    \begin{equation*}
    	\begin{array}{rcccl}
    	a  +  b & = &\frac{1}{5}
    	\\ 
    	\\
    	a & = & \frac{1}{5}  -  b
    	\\
    	\\
    	a & = & \frac{1}{5} - \frac{1}{10}
    	\\
    	\\ 
    	a & = & \frac{1}{10}
    	
    	\end{array}
    \end{equation*}
    Luego los valores de a y b son : $ \frac{1}{10}$
    
    
    
	\section*{Ejercicio 2}
	\paragraph{Pregunta I-b}
	Hallar la funci\'on de probabilidad marginal para cada variable.


	
	$$
		\begin{tabular}{|c|c|c|c|}
			\hline
			X & 2 & 4 & 6
			\\\hline
			P $\left(X=x\right)$ & 0.4 & 0.3 & 0.3
			\\\hline
		\end{tabular}
      	\hspace{1.5cm}
    	\begin{tabular}{|c|c|c|c|}
    		\hline
    		Y & 1 & 3 & 5
    		\\\hline
    		P $\left(Y=y\right)$ & 0.2 & 0.4 & 0.4
    		\\\hline
    	\end{tabular}
    $$
    \paragraph{Pregunta I-c} Diga si X y Y son independientes. Justifique.
    
    \vspace*{0.5cm}

    -Para que X y Y sean independientes tiene que cumplirse:
    \begin{equation*}
    P\left(X=a_i,Y=b_j\right) =  P\left(X=a_i\right) P\left(Y=b_j\right)
	\end{equation*} 
	

	\section*{Ejercicio 3}
     
\end{document}	