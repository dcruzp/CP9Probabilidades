\documentclass[12pt]{article}

\usepackage{amsmath}

\begin{document}
\title{Respuesta de la Clase Pr\'actica 9}
\author{Daniel De La Cruz Prieto}
\date{\today}

\maketitle


\section*{Ejercicio 1}

\subsection*{1-a) Calcular los valores de $a$ y de $b$ si se sabe que $EX =~ 3.8.$}

\begin{equation*}
    P_X\left(2\right) = \sum_{j=1}^{\infty}	P_{\left(X,Y\right)} \left(x,b_j\right) = \sum_{j=1}^{\infty} P\left(X=2,Y=b_j\right)
\end{equation*}

\begin{equation*}
    \begin{array}{rcccccccl}
        P_X\left(2\right) & = & 0.1 & + & 0   & + & 0.3 & = & 0.4
        \\
        P_X\left(4\right) & = & a   & + & 0.2 & + & 0   & = & a + 0.2
        \\
        P_X\left(6\right) & = & 0   & + & 0.2 & + & b   & = & b + 0.2
    \end{array}
\end{equation*}

\begin{equation*}
    EX = \sum_{i=1}^{\infty} x_j P\left(X=x_j\right)
\end{equation*}

\begin{equation*}
    \begin{array}{rcccccl}
        EX            & = & 2(\frac{4}{10})          & + & 4(a+\frac{2}{10})             & + & 6(b+\frac{2}{10})
        \\
        \\
        EX            & = & 2 \left(\frac{2}{5}\right)  & + & 4a + 4 \left(\frac{1}{5}\right))  & + & 6b + 6 \left(\frac{1}{5}\right) 
        \\
        \\
        EX            & = & \frac{4}{5}     & + & 4a + \frac{4}{5}     & + & 6b + \frac{6}{5}
        \\
        \\
        EX            & = & \frac{14}{5}    & + & 4a                   & + & 6b
        \\
        \\
        \Rightarrow 1 & = & 4a              & + & 6b
    \end{array}
\end{equation*}

Se sabe que :

\begin{equation*}
    \sum_{i=1}^{\infty} \sum_{j=1}^{\infty}	P_{\left(X,Y\right)} \left(a_i,b_j\right) = 1
\end{equation*}

\begin{equation*}
    \begin{array}{rcl}
        \sum_{i=1}^{\infty} \sum_{j=1}^{\infty}	P_{\left(X,Y\right)} \left(a_i,b_j\right) & = & 0.1 + 0.3 + a + 0.2 + 0.2 + b
        \\
        1                                                                                & = & 0.8 + a + b
        \\
        1 - 0.8                                                                          & = & a + b
        \\
        0.2                                                                              & = & a + b
        \\
        \frac{1}{5}                                                                      & = & a + b
    \end{array}
\end{equation*}

Podemos plantear el sistema siguiente y obtenener los valores de a y b :

\begin{equation*}
    \begin{array}{rccccl}
        I      & 4a & + & 6b & = & 1
        \\
        \amalg & a  & + & b  & = & \frac{1}{5}
    \end{array}
\end{equation*}

Y sustituyendo en I :

\begin{equation*}
    \begin{array}{rcl}
        4(\frac{1}{5} - b)  +  6b & = & 1
        \\
        \\
        \frac{4}{5} - 4b  +  6b   & = & 1
        \\
        \\
        \frac{4}{5}  +  2b        & = & 1
        \\
        \\
        2b                        & = & 1  -  \frac{4}{5}
        \\
        \\
        2b                        & = & \frac{1}{5}
        \\
        \\
        b                         & = & \frac{1}{10}
    \end{array}
\end{equation*}


Despejando $b = \frac{1}{10}$ en la ecuaci\'on $\amalg$ para hallar a :


\begin{equation*}
    \begin{array}{rcccl}
        a  +  b & = & \frac{1}{5}
        \\
        \\
        a       & = & \frac{1}{5}  -  b
        \\
        \\
        a       & = & \frac{1}{5} - \frac{1}{10}
        \\
        \\
        a       & = & \frac{1}{10}
    \end{array}
\end{equation*}
Luego los valores de a y b son : $ \frac{1}{10}$


\subsection*{1-b) Hallar la funci\'on de probabilidad marginal para cada variable.}


\begin{equation*}
    \begin{tabular}{|c|c|c|c|}
        \hline
        X                    & 2   & 4   & 6
        \\\hline
        P $\left(X=x\right)$ & 0.4 & 0.3 & 0.3
        \\\hline
    \end{tabular}
    \hspace{1.5cm}
    \begin{tabular}{|c|c|c|c|}
        \hline
        Y                    & 1   & 3   & 5
        \\\hline
        P $\left(Y=y\right)$ & 0.2 & 0.4 & 0.4
        \\\hline
    \end{tabular}
\end{equation*}


\subsection*{1-c)Diga si $X$ y $Y$ son independientes. Justifique.}

\vspace*{0.5cm}

Para que X y Y sean independientes tiene que cumplirse:
\begin{equation*}
    \fbox{$
    P\left(X=a_i,Y=b_j\right) =  P\left(X=a_i\right) P\left(Y=b_j\right)
    $}
\end{equation*}

\begin{flushleft}
    Si probamos para el para los sigientes valores que toman las varibles aleatoria  $X = 2$ y $Y=1 $ obtenemos : 
\end{flushleft}
\begin{equation*}
    P\left(2,1\right)  =  \frac{1}{10} \neq \frac{2}{25}   = P\left(X=2\right) P\left(Y=1\right)
\end{equation*} 
  
\begin{flushleft}
    Por lo que si para los valores $ X = 2 $ , $ Y = 1 $ no se cumple la igualdad entonces podemos decir que las variables 
    $X$ y $Y$ son dependientes 
\end{flushleft}


\subsection*{1-d) Calcular $E\left(XY\right)$ }

\begin{equation*}
    \begin{array}{rcl}
        E\left(XY\right) & = & \displaystyle 2 \left(\frac{1}{10}\right) + 10 \left(\frac{3}{10}\right) + 4 \left(\frac{1}{10}\right)
        \\
        \\
        && \displaystyle  +  12 \left(\frac{2}{10}\right) + 18 \left(\frac{2}{10}\right) + 30 \left(\frac{1}{10}\right)
        \\
        \\
        \displaystyle E\left(XY\right) & = & \displaystyle \frac{126}{100} = 12.6
    \end{array}
\end{equation*}


\section*{Ejercicio 2}

\subsection*{2-a)  Determinar la funciones de probabilidad marginales }

\begin{center}
    \begin{tabular}{|c|c|c|c|c|}
        \hline
        X                   & 4   & 6   & 8   & 10
        \\\hline
        $P\left(X=x\right)$ & 0.2 & 0.2 & 0.2 & 0.4
        \\
        \hline
    \end{tabular}

    \vspace*{0.5cm}
    \begin{tabular}{|c|c|c|c|c|}
        \hline
        Y                   & 0   & 1   & 2
        \\\hline
        $P\left(Y=y\right)$ & 0.4 & 0.2 & 0.4
        \\
        \hline
    \end{tabular}
\end{center}

\subsection*{2-b)  Determinar las funci\'on de distribuci\'on conjunta}

\begin{equation*}
    F_{\left(X,Y\right)} \left(x,y\right) = P \left(X \le x , Y \le y\right)
\end{equation*}

\begin{equation*}
    P \left(X \le x , Y \le y\right) =  \sum_{x_i \le x}^{} \sum_{y_j \le y}^{}  P \left(X = x_i , Y = y_i\right)
\end{equation*}

\begin{center}
    \begin{tabular}{|c|l|l|l|l|}
        \hline
        $F\left(x,y\right)$ & 4 &  6 & 8 & 10 
        \\
        \hline
        0 & 0 &  0.1 & 0.3 & 0.4
        \\
        \hline
        1 & 0 &  0.1 & 0.3 & 0.6
        \\
        \hline
        2 & 0.2 &  0.4 & 0.6 & 1
        \\
        \hline
    \end{tabular}
\end{center}


\subsection*{2 -c) Determinar la varianza de Y}

\begin{equation*}
    V\left(Y\right) = E \left[\left(Y - EY\right)^2\right] = EY^2 - \left(EY\right)^2
\end{equation*}

Calculamos $EY^2$ y despu\'es calculamos $EY$ para poder calcular $V\left(Y\right)$
\begin{equation*}
    \begin{array}{rcl}
        EY^2 & = & \displaystyle 0^2 \left(\frac{2}{5} \right)  + 1^2  \left(\frac{1}{5}\right) + 2^2 \left( \frac{2}{5}\right)
        \\
        \\
        EY^2 & = & \displaystyle \frac{9}{5}
        \\
        \\
        EY   & = & \displaystyle 0 \left(\frac{2}{5} \right)  + 1  \left(\frac{1}{5}\right) + 2 \left( \frac{2}{5}\right)
        \\
        \\
        EY   & = & 1
    \end{array}
\end{equation*}

Ahora nos queda que :

\begin{equation*}
    \begin{array}{rcl}
        V\left(Y\right) & = & EY^2 - \left(EY\right)^2
        \\
        \\
        V\left(Y\right) & = & \displaystyle \frac{9}{5} - 1
        \\
        \\
        V\left(Y\right) & = & \displaystyle \frac{4}{5}
        \\
    \end{array}
\end{equation*}

\subsection*{2- d) Determinar la funcion de probabilidad de la v.a $\left(X|Y = 2\right)$ y su valor esperado }

\begin{flushleft}
    Dada la variable aleatoria $ X | Y = 2 $ Podemos encontrar su funcion de probabilidad con la siguiente formula :
    \begin{equation*}
        P \left(X = a_i | Y = b_j\right) = \frac{P\left(X = a_i , Y = b_j\right)}{P \left(Y = b_j\right)}
        \hspace*{1cm}
        \mbox{, con $b_j$  fijo }
    \end{equation*}
    Ahora para obtener la funci\'on de probabilidad de la variable aleatoria $ X|Y =2$
\end{flushleft}

\begin{equation*}
    \begin{array}{rcccc}
        P\left(X = 4 | Y =2 \right)  & = & \displaystyle \frac{P\left(X=4 , Y=2\right)}{P\left(Y =2\right)}  & = & \displaystyle \frac{1}{2}
        \\
        \\
        P\left(X = 6 | Y =2 \right)  & = & \displaystyle \frac{P\left(X=6 , Y=2\right)}{P\left(Y =2\right)}  & = & \displaystyle \frac{1}{4}
        \\
        \\
        P\left(X = 8 | Y =2 \right)  & = & \displaystyle \frac{P\left(X=8 , Y=2\right)}{P\left(Y =2\right)}  & = & 0
        \\
        \\
        P\left(X = 10 | Y =2 \right) & = & \displaystyle \frac{P\left(X=10 , Y=2\right)}{P\left(Y =2\right)} & = & \displaystyle \frac{1}{4}
    \end{array}
\end{equation*}


\begin{center}
    \renewcommand{\arraystretch}{1.5}
    \begin{tabular}{|c|c|c|c|c|}
        \hline
        X                                    & 4              & 6              & 8   & 10
        \\
        \hline
        $\displaystyle P \left(X|Y=2\right)$ & $ \frac{1}{2}$ & $ \frac{1}{4}$ & $0$ & $ \frac{1}{4}$
        \\
        \hline
    \end{tabular}
\end{center}


\begin{flushleft}
    Vamos ahora a calcular el valor esperado de $X|Y=2$
\end{flushleft}

\begin{equation*}
    \begin{array}{rcl}
        \displaystyle E\left(X|Y=2\right) & = & \displaystyle 4 \left(\frac{1}{2}\right) + 6 \left(\frac{1}{4}\right) + 10\left(\frac{1}{4}\right)
        \\
        \\
        \displaystyle E\left(X|Y=2\right) & = & 6
    \end{array}
\end{equation*}


\subsection*{2-e)  Determinar si X y Y  son independientes}

\begin{flushleft}
    Para determinar si son independientes tenemos que comprobar que se coumple :


    \begin{equation*}
        P \left(X = a_i , Y = b_j\right)  = P \left(X = a_i\right) P \left(Y= b_j\right)
        \hspace*{1cm }
        \mbox{$\forall$ $a_i$,$b_j$ admisibles}
    \end{equation*}


    Ahora si comprobamos para $X = 4 $ , $Y = 0$


    \begin{equation*}
        P \left(X =4 , Y =0 \right) = 0 \neq \left(\frac{1}{5}\right) \left(\frac{2}{5}\right)  = P\left(X =4 \right)  P \left(Y = 0\right)
    \end{equation*}


    Como existe un par de valores para los cuales la igualdad no se cumple entonces
    podemos decir que las variables $X$ y $Y$ no son independientes
\end{flushleft}



\section*{Ejercicio 3}


\begin{center}
    \renewcommand{\arraystretch}{1.5}
    \begin{tabular}{|c|c|c|c|c|}
        \hline
        Y/X & o               & 1               & 2              & 3
        \\
        \hline
        0   & $\frac{4}{81}$  & $\frac{10}{81}$ & $\frac{8}{81}$ & $\frac{2}{81} $
        \\
        \hline
        1   & $\frac{10}{81}$ & $\frac{17}{81}$ & $\frac{8}{81}$ & $\frac{1}{81}$
        \\
        \hline
        2   & $\frac{8}{81}$  & $\frac{8}{81}$  & $\frac{2}{81}$ & 0
        \\
        \hline
        3   & $\frac{2}{81}$  & $\frac{1}{81}$  & 0              & 0
        \\
        \hline
    \end{tabular}
\end{center}

\subsection*{ 3-a) Determinar la probabilidad marginal y de distribuci\'on de Y.}

\begin{equation*}
    P_Y\left(b_j\right) =  \sum_{j=1}^{\infty} P_{\left(X,Y\right)} \left(a_i, b_j\right)
                             =  \sum_{j=1}^{\infty} P\left(X=a_i,Y=b_j\right)
\end{equation*}

\begin{equation*}
    \begin{array}{rclcl}
        \\
        \displaystyle P\left(X = a_i\right) & = & \displaystyle \sum_{j=1}^{\infty} P \left(X = a_i,Y = b_j\right)
        \\
        \\
        \displaystyle P\left(X = 0\right)   & = & \displaystyle \left( \frac{4}{81}\right) + \left(\frac{10}{81}\right) + \left(\frac{8}{81}\right) +\left( \frac{2}{81}\right)     & = &  \displaystyle \frac{24}{81}
        \\
        \\
        \displaystyle  P\left(X = 1\right)   & = & \displaystyle \left(\frac{10}{81}\right) + \left(\frac{17}{81}\right) + \left(\frac{8}{81}\right)  + \left(\frac{1}{81}\right)   & = & \displaystyle \frac{36}{81}
        \\
        \\
        \displaystyle P\left(X = 2\right)   & = & \displaystyle \left(\frac{8}{81}\right) + \left(\frac{8}{81}\right) + \left(\frac{2}{81}\right)                                           & = & \displaystyle \frac{18}{81}
        \\
        \\
        \displaystyle P\left(X = 3\right)   & = & \displaystyle  \left(\frac{2}{81}\right) + \left(\frac{1}{81}\right)                                                                 & = & \displaystyle \frac{3}{81}
    \end{array}
\end{equation*}


\begin{center}
    \renewcommand{\arraystretch}{1.5}
    \begin{tabular}{|c|c|c|c|c|}
        \hline
        Y                   & 0               & 1               & 2               & 3
        \\
        \hline
        $P\left(Y=y\right)$ & $\frac{24}{81}$ & $\frac{36}{81}$ & $\frac{18}{81}$ & $\frac{3}{81}$
        \\
        \hline
    \end{tabular}
\end{center}


Entonces luego :

\begin{equation*}
    \begin{array}{rcccl}
        F_Y\left(y\right) & = & P\left(Y \leq y\right)
                         & = & \displaystyle \sum_{i\leq y} P\left(Y=i\right)
        \\
        \\
        F_Y\left(0\right) & = & P\left(Y\leq0\right)        & = & \displaystyle \frac{24}{81}
        \\
        \\
        F_Y\left(1\right) & = & P\left(Y\leq1\right)        & = & \displaystyle \frac{24}{81} + \frac{36}{81} = \frac{60}{81}
        \\
        \\
        F_Y\left(2\right) & = & P\left(Y\leq2\right)        & = & \displaystyle \frac{24}{81} + \frac{36}{81} + \frac{18}{81} = \frac{78}{81}
        \\
        \\
        F_Y\left(3\right) & = & P\left(Y\leq3\right)        & = & \displaystyle \frac{24}{81} + \frac{36}{81} + \frac{18}{81} + \frac{3}{81}  = 1
    \end{array}
\end{equation*}

\begin{flushleft}
    Entonces la funci\'on de distribuci\'on es : 
\end{flushleft}

\begin{equation*}
    F\left(Y\right) = \begin{cases}
        0           & \mbox{si   $y < 0 $}
        \\
        \\
        \displaystyle \frac{24}{81} & \mbox{   si   $0\leq y < 1$}
        \\
        \\
        \displaystyle \frac{60}{81} & \mbox{   si   $1\leq y < 2$}
        \\
        \\
        \displaystyle \frac{78}{81} & \mbox{   si   $2\leq y < 3$}
        \\
        \\
        1           & \mbox{     si   $y \geq 3$}
    \end{cases}
\end{equation*}

\subsection*{ 3-b ) Determinar el valor esperado de $ X|Y = 1$.}

Usando la  siguinte expreci\'on:


\begin{equation*}
    \fbox{$
    \displaystyle P\left(X = a_i|Y = b_j\right)  =  \frac{P\left(X = a_i,Y = b_j\right)}{P\left(Y = b_j\right)}
    $}
\end{equation*}

\begin{equation*}
    \begin{array}{rcl}
        P\left(X = 0|Y = 1\right) & = & \displaystyle \frac{10}{36}
        \\
        \\
        P\left(X = 1|Y = 1\right) & = & \displaystyle \frac{17}{36}
        \\
        \\
        P\left(X = 2|Y = 1\right) & = & \displaystyle \frac{8}{36}
        \\
        \\
        P\left(X = 3|Y = 1\right) & = & \displaystyle \frac{1}{36}
    \end{array}
\end{equation*}


\begin{center}
    \renewcommand{\arraystretch}{1.5}
    \begin{tabular}{|c|c|c|c|c|}
        \hline
        $X|Y = 1$                      & 0               & 1               & 2              & 3
        \\
        \hline
        $P\left(X = a_i| Y = 1\right)$ & $\frac{10}{36}$ & $\frac{17}{36}$ & $\frac{8}{36}$ & $\frac{1}{36}$
        \\
        \hline
    \end{tabular}
\end{center}


Por lo tanto :

\begin{equation*}
    \begin{array}{rcl}
        E\left(X|Y = 1\right) & = & \displaystyle 0  \left(\frac{10}{36}\right) + 1 \left(\frac{17}{36} \right) + 2 \left(\frac{8}{36}\right)  + 3 \left( \frac{1}{36} \right)
        \\
        \\
        E\left(X|Y = 1\right) & = & 1
    \end{array}
\end{equation*}

\subsection*{3-c) Determinar la funci\'on de distribuci\'on $ X | Y > 0.$}

    \begin{equation*}
        \fbox{$
        \displaystyle P(X|Y>0)=\frac{P(X=a_i,Y>0)}{P(Y>0)}
        $}
    \end{equation*}
  
    \begin{center}
        \renewcommand{\arraystretch}{1.5}
        \begin{tabular}{|c|c|c|c|c|}
            \hline
            $x$ & 0 & 1 & 2& 3\\ \hline
            $X|Y>0$ & $\frac{20}{57}$ & $\frac{26}{57}$ & $\frac{10}{57}$ & $\frac{1}{57}$\\ \hline
        \end{tabular}
    \end{center}
    

    \begin{flushleft}
        Entonces la funcion de distribuci\'on me queda : 
    \end{flushleft}

    \begin{center}
        \renewcommand{\arraystretch}{1.5}
        \begin{tabular}{|c|c|c|c|c|c|}
                \hline
                $x$ & $x<0$ & $0\leq x < 1$ & $1 \leq x< 2$ & $2 \leq x<3$& $x \geq 3$\\ \hline
                $F_{X|Y>0}(x)$ & 0 & $\frac{20}{57}$ & $\frac{46}{57}$ & $\frac{56}{57}$ & 1\\
                \hline
        \end{tabular}
    \end{center}

\subsection*{3-d) Determinar si $X$ y $Y$ son independientes}

Usando la siguiente expreci\'on :


\begin{equation}
    \fbox{$
    \displaystyle P\left(X = a_i,Y = b_j\right)  =  P\left(X = a_i\right) P\left(Y = b_j\right)
    $}
\end{equation}

\begin{flushleft}
    Ahora calculamos la probabilidad marginal de $X$
\end{flushleft}

\begin{equation*}
    \displaystyle P\left(X = a_i\right)  =  \displaystyle \sum_{j=1}^{\infty} P \left(X = a_i,Y = b_j\right)
\end{equation*}

\begin{equation*}
    \begin{array}{rclcl}
        \displaystyle P\left(X = 0\right)   & = & \displaystyle \left( \frac{4}{81}\right) + \left(\frac{10}{81}\right) + \left(\frac{8}{81}\right) +\left( \frac{2}{81}\right)     & = &  \displaystyle \frac{24}{81}
        \\
        \\
        \displaystyle  P\left(X = 1\right)   & = & \displaystyle \left(\frac{10}{81}\right) + \left(\frac{17}{81}\right) + \left(\frac{8}{81}\right)  + \left(\frac{1}{81}\right)   & = & \displaystyle \frac{36}{81}
        \\
        \\
        \displaystyle P\left(X = 2\right)   & = & \displaystyle \left(\frac{8}{81}\right) + \left(\frac{8}{81}\right) + \left(\frac{2}{81}\right)                                           & = & \displaystyle \frac{18}{81}
        \\
        \\
        \displaystyle P\left(X = 3\right)   & = & \displaystyle  \left(\frac{2}{81}\right) + \left(\frac{1}{81}\right)                                                                 & = & \displaystyle \frac{3}{81}
    \end{array}
\end{equation*}

\begin{flushleft}

Como la tabla es sim\'etrica entonces la variables aleatoria $Y$ tomo los mismos valores que la varaible $X$
por lo tanto no es necesario volver a  calcular los valores 
Luego chequeamos que se cumple (1) para todo valor admisible de las varaibles $X$ y $Y$ Comprobando :  
\end{flushleft}


\begin{equation*}
    P\left(X = 0,Y = 0\right) =   \displaystyle \frac{4}{81}  \neq \displaystyle \left(\frac{24}{81}\right)^2 = P\left(X = 0\right) P\left(Y = 0\right)
\end{equation*}

\begin{flushleft}
    La igualdad (1) no se cumple para los valores  $X = 0,Y = 0$ .Por lo que concluimos que 
    las varaibles son dependientes 
\end{flushleft}


\section*{Ejercicio 4 }

\begin{itemize}
    \item $X$: N\'umero de caras en las tres tiradas
    \item $Y$: Difereencia en valor absoluto entre el n\'umero de caras y el de escudos
\end{itemize}

\subsection*{4-a) Determina la funci\'on de probabilidad conjunta del vector $\left(X,Y\right)$}


\begin{center}
    \renewcommand{\arraystretch}{1.5}
    \begin{tabular}{|c|c|c|c|c|}
        \hline
        $Y|X$ & 0             & 1               & 2              & 3
        \\
        \hline
        1     & 0             & $  \frac{3}{8}$ & $\frac{3}{8} $ & 0
        \\
        \hline
        3     & $\frac{1}{8}$ & 0               & 0              & $\frac{1}{8}$
        \\
        \hline
    \end{tabular}
\end{center}


\subsection*{4-b) Determinar la media y al desviaci\'on t\'ipica de las funciones marginales  de $X$ y  $Y$ }

\begin{equation*}
    P_x \left(a_i\right) = \sum_{j=i}^{\infty} P_{\left(X,Y\right)}\left(a_i,b_j\right) = \sum_{j=1}^{\infty} P \left(X =a_i , Y = b_j\right)
\end{equation*}


\begin{center}
    \renewcommand{\arraystretch}{1.5}
    \begin{tabular}{|c|c|c|c|c|}
        \hline
        $X$                   & 0             & 1             & 2             & 3
        \\
        \hline
        $P_X\left(X=x\right)$ & $\frac{1}{8}$ & $\frac{3}{8}$ & $\frac{3}{8}$ & $\frac{1}{8}$
        \\
        \hline
    \end{tabular}
\end{center}


\begin{center}
    \renewcommand{\arraystretch}{1.5}
    \begin{tabular}{|c|c|c|}
        \hline
        $Y$                   & 1             & 3
        \\
        \hline
        $P_Y\left(Y=y\right)$ & $\frac{6}{8}$ & $\frac{2}{8}$
        \\
        \hline
    \end{tabular}
\end{center}


\begin{flushleft}
    Para encontrar la funci\'on de distribuci\'on tenemos que :
\end{flushleft}



\begin{equation*}
    \fbox{$\displaystyle
            F_X\left(x\right) = P \left(X <x\right) =  \sum_{a_i\le x} P \left(X = a_i\right)
        $}
\end{equation*}


\begin{flushleft}
    Para encontrar la funci\'on de distribuci\'on de la variable aleatoria $X$
\end{flushleft}


\begin{equation*}
    \begin{array}{cp{1cm}c}


        \begin{array}{rcl}
            F_X\left(0\right) & = & \frac{1}{8}
            \\
            \\
            F_X\left(1\right) & = & \frac{4}{8}
        \end{array}
         &
         &
        \begin{array}{rcl}
            F_X\left(2\right) & = & \frac{7}{8}
            \\
            \\
            F_X\left(3\right) & = & 1
        \end{array}
    \end{array}
\end{equation*}


\begin{equation*}
    F\left(X\right) = \begin{cases}
        0           & \mbox{si   $x < 0 $}
        \\
        \frac{1}{8} & \mbox{   si   $0\le x < 1$}
        \\
        \frac{4}{8} & \mbox{   si   $1\le x < 2$}
        \\
        \frac{7}{8} & \mbox{   si   $2\le x < 3$}
        \\
        1           & \mbox{     si   $x>3$}
    \end{cases}
\end{equation*}


\begin{flushleft}
    Ahora para encontrar la funci\'on de distribuci\'on de la variable aleatoria $Y$
\end{flushleft}


\begin{equation*}
    \begin{array}{cp{1cm}c}
        \begin{array}{ccc}
            F_{Y} \left(1\right) & = & \displaystyle \frac{6}{8}
        \end{array}
         &
         &
        \begin{array}{ccc}
            F_{Y} \left(3\right) & = & 1
        \end{array}
    \end{array}
\end{equation*}

\begin{equation*}
    F\left(Y\right) = \begin{cases}
        0           & \mbox{ si   $y < 1 $}
        \\
        \frac{6}{8} & \mbox{ si   $1\leq y < 3$}
        \\
        1           & \mbox{ si   $y\geq 3$}
    \end{cases}
\end{equation*}

\begin{flushleft}
    Ahora calculamos la media de $X$ y de $Y$
\end{flushleft}

\begin{flushleft}
    Para la variable $X$ tenemos :
\end{flushleft}

\begin{equation*}
    \begin{array}{rcl}
        EX & = & \displaystyle 0 \left(\frac{1}{8}\right) + 1 \left(\frac{3}{8}\right) + 2 \left(\frac{3}{8}\right) + 3 \left(\frac{1}{8}\right)
        \\
        \\
        EX & = & \displaystyle \frac{3}{2}
    \end{array}
\end{equation*}

\begin{flushleft}
    Ahora para la variable $Y$ tenemos :
\end{flushleft}

\begin{equation*}
    \begin{array}{rcl}
        EY & = & \displaystyle 1 \left(\frac{6}{8}\right) + 3 \left(\frac{2}{8}\right)
        \\
        \\
        EY & = & \displaystyle \frac{3}{2}
    \end{array}
\end{equation*}

\begin{flushleft}
    Para calcular la desviaci\'on primero tenemos que calcular la varianza
\end{flushleft}

\begin{equation*}
    \fbox{$V\left(X\right) = EX^2 - \left(EX\right)^2
        $}
\end{equation*}

\begin{flushleft}
    $\Rightarrow$ Para calcular la varianza de la varaible $X$ tenemos :
\end{flushleft}

\begin{equation*}
    \begin{array}{rcl}
        EX^2 & = & \displaystyle 0^2 \left(\frac{1}{8}\right) + 1^2 \left(\frac{3}{8}\right) + 2^2 \left(\frac{3}{8}\right) + 3^2 \left(\frac{1}{8}\right)
        \\
        \\
        EX^2 & = & 3
    \end{array}
\end{equation*}

\begin{flushleft}
    Si sistituimos los valores que calculamos ante tenemos que:
\end{flushleft}

\begin{equation*}
    \fbox{$V\left(X\right) = \frac{3}{4}$}
\end{equation*}

\begin{flushleft}
    $\Rightarrow$ Para calcular la varianza de la varaible $Y$ tenemos :
\end{flushleft}

\begin{equation*}
    \begin{array}{rcl}
        EY^2 & = & \displaystyle 1^2 \left(\frac{6}{8}\right) + 3^2 \left(\frac{2}{8}\right)
        \\
        \\
        EY^2 & = & 3
    \end{array}
\end{equation*}

\begin{flushleft}
    Si sistituimos los valores que calculamos ante tenemos que:
\end{flushleft}

\begin{equation*}
    \fbox{$V\left(Y\right) = \frac{3}{4}$}
\end{equation*}

\begin{flushleft}
    Para calcular la desviacion estandar o t\'ipica tenemos que :
\end{flushleft}

\begin{equation*}
    \fbox{$\sigma \left(X\right) = \sqrt{V\left(X\right)} $}
\end{equation*}

\begin{flushleft}
    Nos queda entonces que la desviaci\'on estandar o tipica para las variables $X$ y $Y$
    es la misma :
\end{flushleft}

\begin{equation*}
    \displaystyle \sigma \left(X \right) = \sigma\left(Y\right)  = \frac{\sqrt{3}}{2}
\end{equation*}

\subsection*{4-c) Determinar la covarianza y el coeficiente de correlaci\'on }

    \begin{equation*}
        \begin{array}{rcl}
            E\left[XY\right]& = & \displaystyle \sum_{j=i}^{\infty}\sum_{i=1}^{\infty}g(b_j,a_i)P(Y=b_j,X=a_i)
            \\
            \\
            & = &\displaystyle \sum_{j=i}^{\infty}\sum_{i=1}^{\infty}b_ja_iP(Y=b_j,X=a_i)
            \\
            \\
            &=& \displaystyle \frac{18}{8}
        \end{array}
    \end{equation*}
    
    \begin{flushleft}
        La covarianza es : 
    \end{flushleft}
    \begin{equation*}
        Cov(X,Y)=E[XY]-EXEY=\frac{18}{8}-\frac{9}{4}=0
    \end{equation*}
   
    \begin{flushleft}
        Y el coeficiente de correlaci\'on  es : 
    \end{flushleft}

    \begin{equation*}
        \rho=\frac{Cov(X,Y)}{\sqrt{V(X)V(Y)}}=\frac{0}{\sqrt{\left(\frac{3}{4}\right)^2}}=0
    \end{equation*}
    
\end{document}